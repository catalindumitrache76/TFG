\begin{center}

{\Large \bfseries RESUMEN EJECUTIVO}

\vspace{1.5cm}
\end{center}


Este proyecto se centra en presentar y demostrar un modelo de integración de información relativa a productos fitosanitarios y su aplicación en diferentes productos agrícolas provenientes de fuentes heterogéneas en un esquema único y escalable. \par
Para lograr esto, se han seleccionado diferentes fuentes de datos sobre productos fitosanitarios, presentes en distintos formatos y se han analizado varias tecnologías de integración para elegir el stack tecnológico que más se adecue al problema en cuestión.\par
Así pues, en este proyecto se han usado herramientas de almacenamiento elegidas bajo un criterio de escalabilidad futura y herramientas de procesado de datos bajo el criterio de proporcionar soporte al mayor abanico de fuentes heterogéneas posibles. \textit{Apache Hadoop} fue el indicado como sistema responsable del almacenamiento en conjunto con \textit{Apache Hive} para facilitar el acceso a los datos mediante una aproximación relacional. Para el procesado de los datos se ha empleado \textit{Talend Big Data}.\par
Para demostrar la viabilidad del sistema, se ha desarrollado un prototipo funcional que recoge datos de los productos fitosanitarios de España y Europa y complementa la información de ambas fuentes, constituyendo un primer paso hacia ese modelo compartido donde varias fuentes heterogéneas concuerdan en un mismo esquema congruente. Los datos se pueden ver mediante una aplicación web desarrollada con \textit{JHipster}, un generador de proyectos ligero sobre Java capaz de desplegar rápidamente una aplicación con una cuidada GUI.  \\\par

\textbf{Palabras clave:}  Productos fitosanitarios, Hadoop, Hive, Integración, ETL, Base de datos, JHipster, MySQL, Modelo único, BigData, Escalabilidad, Sqoop, Fuentes heterogéneas.
