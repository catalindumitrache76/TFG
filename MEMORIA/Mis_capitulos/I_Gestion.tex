\chapter{Gestión} \label{gestion}
Este capítulo engloba aquellos aspectos que tienen que ver diréctamente con la gestión del proyecto: la metodología que se ha seguido a lo largo de todo su desarrollo, la  recogida de los esfuerzos y la organización del trabajo así como el control de versiones de los diferentes componentes del proyecto. Se reserva un apartado para mencionar las diferentes pautas e imposiciones que han afectado a un desarrollo libre del proyecto y  se recoge un estudio acerca de la estimación del coste del proyecto en sí. 

\section{Metodología} \label{gestion.metodologia}
Este apartado explica la metodología que se ha seguido para llevar a cabo el desarrollo del proyecto, desde su fase inicial llamada prueba de concepto (Sección \ref{implementacion.prueba}) hasta su finalización. Se podrían distinguir dos fases conceptuales acordes a los diferentes \textit{modus operandi} del desarrollo: 
\paragraph*{Prueba de concepto.} El desarrollo de la prueba de concepto ha sido guiado por el director del proyecto. Esto es, el director del proyecto marcó inicialmente el panorama global y el diseño que se quería seguir. A partir de alli, el trabajo del alumno fue instalar y configurar las herramientas propuestas por el director y validar su integración mediante un flujo de datos de prueba. Si tras un período de pruebas exhaustivas algún componente fallaba o no cumplía con los requisitos especificados en el análisis del proyecto, dicho componente era desechado y se buscaba una alternativa viable al mismo. Lo mismo ocurría si alguna conexión de integración concreta fallaba; por ejemplo, en el caso de intentar conectar \textit{Hive} con \textit{JHipster} diréctamente, al comprobar que era una solución inviable que no cumplía con los requisitos del proyecto se decidió que la alternativa sería mantener la base de datos original de \textit{JHipster} (\textit{MySQL}) y agregar un componente intermedio (\textit{Sqoop}) para la transferencia de los datos desde \textit{Hive} a \textit{MySQL}.
\paragraph*{Prototipo real.} Una vez integradas las diferentes herramientas y elegido el \textit{stack tecnológico} final, la implementación del prototipo real se ha llevado a cabo mediante una metodología iterativa. Teniendo en mente que se trata de un proyecto con un potencial de crecimiento casi ilimitado, la aproximación más lógica fue agregar valor al proyecto mediante una aproximación de iteraciones agregativas, empezando por la integración de los productos fitosanitarios autorizados de España y siguiendo por los datos acerca sustancias activas de la base de datos Europea sobre pesticidas.  Así pues, en la primera iteración se diseñó e implementó el flujo capaz de descargar los datos acerca de los productos fitosanitarios autorizados de España de la fuente, almacenarlos en \textit{Hadoop}, transformarlos y prepararlos para su inserción en \textit{Hive}, su transferencia a \textit{MySQL} y su posterior visualización en \textit{JHipster}. En una segunda iteración se realizó lo mismo pero con los datos de las sustancias activas extraídos de la base de datos Europea de pesticidas. Siguiendo esta metodología y con el objetivo fijado en el crecimiento del proyecto se puede observar que los futuros avances del sistema se pueden realizar de la misma manera. Otro desarrollador podría retomar el trabajo en este punto y hacer que el programa siga creciendo mediante la expansión del número de iteraciones que agreguen nuevos flujos de datos para dar sporte a nuevas fuentes. La tercera iteración supuso la agregación del soporte capaz de mapear los datos de la primera iteración con los de la segunda, en una versión más que nada ilustrativa; a pesar de que dicha integración no es capaz de mapear el 100\% de los datos, esto no es un problema peusto que no era el objetivo perseguido. Lo que se perseguía era validar el modelo de integración y dar soporte a un crecimiento sencillo de la solución. Por último, siguiendo esta filosofía de iteraciones se agregó en una cuarta iteración un mecanismo para la detección de errores o inconsistencias en los datos integrados provenientes de diferentes fuentes. 

\section{Pautas e imposiciones} \label{gestion.pautas}
La gestión y desarrollo del proyecto, en todas sus fases se ha visto restringida por diferentes pautas, imposiciones o recomendaciones provenientes de terceras partes. Este apartado pretende aclarar algunas de estas cuestiones para reflejar aquellas decisiones que han condicionado, para bien o para mal, el desenvolvimiento del alumno. 
\par Desde el inicio del proyecto el director impuso algunas de las herramientas a utilizar, así como el diseño a priori de la solución. \textit{Hadoop}, \textit{Hive} y \textit{JHipster} fueron el \textit{core} tecnológico que el director estableció para la realización del proyecto. Como primer diseño, además, el director expuso un modelo en el que los datos tanto procesados como sin procesar serían almacenados en \textit{Hadoop}, consumidos desde \textit{Hive} e importados diréctamente a \textit{JHipster}, sustituyendo la base de datos de \textit{JHipster} por \textit{Hive}. Tras observar que este modelo no cumplía con los requisitos del proyecto, se optó por la otra variante, mediante \textit{Apache Sqoop}, tal como se ha mencionado anteriormente. 
\par Otra de las herramientas recomendadas por el director del proyecto fue \textit{Pentaho Kettle} y, como se puede observar en la sección \ref{implementacion.problemas}, fue una de las piezas que más problemas acabó dando. Ante esta situación, el director recomendó \textit{Talend}, que resultó ser un mejor componente y que satisfacía con los requisitos de la fase de análisis.
\par Otro aspecto que se debe tener en cuenta es que el proyecto se trata de un \textit{TFG} y no de una solución comercial. Por ello, hay unas normas o pautas  establecidas que delimitan y guían en el desarrollo del mismo: la limitación de las horas de dedicación recomendadas, que se corresponden a los 12 créditos ECT, la inclusión de una memoria suficientemente extensa para recopilar todos los aspectos del desarrollo del proyecto e incluso la limitación económica implícita, esto es, no existe una remuneración monetaria para el alumno tras el desarrollo del proyecto. 

\section{Organización y control de versiones} \label{gestion.organizacion}
Otro área de la gestión del proyecto es su organización, a través de sus diferentes componentes. En este apartado se pretende dar una visión global de las estructuras y tecnologías involucradas en la organización del proyecto. 
\par En primer lugar, cabe mencionar que las diferentes herramientas que constituyen el \textit{core }tecnológico del proyecto (\textit{Hadoop}, \textit{Hive}, \textit{Sqoop}, \textit{JHipster}, \textit{Talend}, \textit{MySQL}) se han instalado sobre el equipo del alumno, en una partición local del disco duro. Esto proporcionó rapidez de despliegue y desarrollo para el alumno pero podría suponer dificultades a la hora de expandir el proyecto e incluso riesgos adicionales debido a una inexistencia de tolerancia a fallos o copias de seguridad. No obstante, tratándose de un \textit{TFG} se asumieron los riesgos y se adoptó esta postura como la más adecuada. 
\par Otro aspecto de la organización se centra en la aplicación desarrollada con \textit{JHipster}. Inicialmente, esta se instaló al igual que las herramientas anteriores en el equipo local del alumno. No obstante, dado que sería una pieza fundamental y sobre la que se desarrollaría el software en sí, se decidió subirla a \textit{GIT} para mantener un control de versiones sobre ella. Profundizando más acerca de la organización del software desarrollado, dentro de la aplicación de \textit{JHipster} se creó un paquete encargado de mantener todo el código desarrollado por el alumno. Este paquete, llamado \textit{processes}, junto con sus subpaquetes y clases se puede observar en la figura \ref{fig:diag_clases}. Existe una clase principal llamada \textit{Schedule} encargada de lanzar los diferentes procesos. Aparte, se han designado distintos subpaquetes en función de las herramientas contra las que atacan: el paquete \textit{talend} es el que contiene los métodos encargados de ejecutar los trabajos desarrollados con \textit{Talend}. El paquete \textit{sqoop} contiene los métodos necesarios para poner en marcha una transferencia de \textit{Sqoop} desde \textit{Hive} a \textit{MySQL}. El paquete \textit{hive} contiene métodos que atacan contra la base de datos de \textit{Hive} mientras que el paquete \textit{mysql} contiene métodos que atacan contra la base de datos de \textit{MySQL}. El paquete \textit{common\_methods} es el único especial y contiene métodos públicos que puedan ser usados desde cualquiera de los demás paquetes. 
\par Como esta memoria también se quería mantener bajo un control de versiones riguroso, también se decidió que debería formar parte del software subido a \textit{GIT}. Así pues, en la carpeta raíz del proyecto de \textit{JHipster} se creó una carpeta llamada \textit{MEMORIA} donde se almacenaba todo lo referente a esta memoria.
\par Por último lugar, lo único restante de los diferentes componentes del proyecto son los diagramas desarrollados por el alumno tanto para el diseño de la aplicación como para los diferentes capítulos de la memoria y las hojas de gestión de esfuerzos. Estos componentes se crearon en \textit{Google Drive} y se han ido actualizando allí mismo. Dado que el alumno usa la aplicación web \textit{draw.io} propietaria de \textit{Google Drive} para realizar los diagramas, esta aproximación se consideró como la más adecuada. 

\section{Control de esfuerzos} \label{gestion.esfuerzos}
Explicar como se ha hecho el control de esfuerzos. Dónde están las hojas, etc. Hacer un resúmen de los esfuerzos y estadísticas contabilizadas del proyecto, mencionando aquellos puntos de más interés. 

\section{Estimación del coste} \label{gestion.estimacion}
¿Cuanto cuesta en dinero tu TFG? - Revisar el trabajo de Cabrera.