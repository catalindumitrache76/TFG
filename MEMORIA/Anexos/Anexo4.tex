\chapter{Gestión} \label{d.gestion}
\section{Cálculo del coste de la hora de trabajo del alumno} \label{d.gestion.estimacion}
A continuación se exponen los cálculos desglosados y en detalle para calcular el coste de la hora de trabajo del alumno. Para conseguir el precio por hora para un salario de 1,500€ hacemos lo siguiente:
\begin{enumerate}
\item Calculamos el salario bruto anual:

Salario mensual × 12 meses

1,500.00 × 12 = \textbf{18,000€}
\item Hallamos el precio básico por hora:

Salario anual ÷ 2,080 horas posibles (8 horas × 5 días × 52 semanas)

18,000.00 ÷ 2,080 = \textbf{8.65€}
\item Sumamos el precio de las horas que no trabajarás:

14 días festivos × 8 hs. = 112 hs. +

21 días de vacaciones × 8 = 168 hs. +

7 días de incapacidad × 8 hs. = 56 hs.

\textbf{336 hs. × 8.65 de coste básico = 2,907.69€}
\item Calculamos el valor del tiempo que dedicas a presupuestos, reuniones, venta, formación... (tiempo administrativo):

50\% (2,080 hs. posibles - 336 horas que no trabajarás) × 8.65 coste básico.

\textbf{50\% × 1,744 hs. × 8.65 = 7,546.15€}
\item Calculamos el total de tus gastos fijos:

Alquiler mensual: 100.00 +

Servicios mensuales: 50.00 +

Autónomos mensual: 260.00 +

Otros gastos fijos mensuales: 50.00

Gastos fijos mensuales 460.00 × 12 meses = \textbf{5,520€ fijos anuales}
\item Sumamos el valor de las horas que no trabajas más el valor del tiempo de administración y el de los gastos fijos para obtener el precio extra anual por tu trabajo.

2,907.69 + 7,546.15 + 5,520.00 = \textbf{15,973.85€ de precio extra anual}
\item Ahora calculamos lo que ganarás al año:

2,080 hs. posibles al año - 336 hs. de vacaciones, festivos e incapacidad - 872 hs. de reuniones y presupuestos × 8.65 coste básico

\textbf{872 horas de trabajo × 8.65 = 7,546.15€ de beneficio anual.}
\item Calculamos el porcentaje de rentabilidad dividiendo el coste entre el beneficio:

15,973.85 ÷ 7,546.15 = 211.682\%

\item Por último para calcular la hora de trabajo sumamos el porcentaje de rentabilidad y el porcentaje de beneficio deseado a nuestro precio básico:

8.65 + 8.65 × 211.682\% + 8.65 × 20\% = \textbf{28.70€ por hora de trabajo}.
\end{enumerate}















