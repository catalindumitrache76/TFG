\chapter{Clientes} \label{b.clientes}
\section{Clientes potenciales} \label{b.clientes.potenciales}
\begin{itemize}
\item \textbf{\textit{aGROSLab}} \cite{agroslab}
\begin{itemize}
\item Registro de Explotaciones - Las parcelas se presentan para su selección en un innovador formato de celdas, facilitando la visualización de todas sus características (identificación, cultivo, superficie, …), el acceso al visor GIS, la aplicación de filtros y su selección individual o en bloque.
\item Registro de Parcelas Agrícolas - permite cargar las parcelas que componen su explotación a partir de la información generada por el aplicativo de gestión de la Solicitud de Ayudas PAC, para la mayoría de las Comunidades Autónomas.
\item Compras de Productos Fitosanitarios - incorpora un registro de compras de productos fitosanitarios, en el que podrá archivar todas sus facturas de compra en formato PDF y a partir del cual podrá registrar los tratamientos realizados.
\item Registro de \gls{tratamiento}- filtra los productos autorizados para cada uno de sus cultivos, presenta las plagas para las que puede ser aplicado según la nomenclatura del MAGRAMA y le informa del tipo y rango de dosis que puede ser aplicada.
\item Registro de Comercialización de Cosecha - facilita el registro de la comercialización de su cosecha, presentado en formato de celdas el conjunto de parcelas de su explotación con un determinado cultivo e informando gráficamente de aquellas en las que puede existir un problema con los plazos de seguridad de un tratamiento fitosanitario.
\item Receta Fitosanitaria 
\item Visor GIS con Capas - permite al agricultor visualizar gráficamente las parcelas que componen su explotación y la información de cultivos y los tratamientos realizados. Una herramienta especialmente útil a la hora de identificar sus diferentes parcelas y tener una visión de conjunto de toda su explotación.
\item Unidades Homogéneas de Cultivo
\item Control de Consumos de Fitosanitarios - permite llevar el control de los productos (fitosanitarios y fertilizantes) adquiridos y los aplicados
\item Importaciones y Exportaciones 
\item Importaciones y Exportaciones 
\end{itemize}
\item \textbf{\textit{Agricolum}} \cite{agricolum}
\begin{itemize}
\item Web + APP móvil y tableta
\item Validación dosis cuaderno de campo
\item Gestión de personal y maquinaria
\item Informes personalizados y oficiales
\item Control por GPS
\item Control stock
\item Gestión económica
\item Rendimientos por campos y cultivos
\item Soporte telefónico y por internet
\item Gestión del cuaderno de campo
\item Aplicación conectada con los datos del Sigpac y fitosanitarios MAGRAMA
\item Sincronización de la información desde cualquier dispositivo
\item Vista de tiempo actual y previsión semanal
\item Ver histórico de todas las tareas realizadas
\item Saber en tiempo real el precio del mercado
\item Exportación de la información en otros formatos
\item Importación de los datos de la PAC
\item Generación del cuaderno de explotación oficial
\end{itemize}
\item \textbf{\textit{Cuaderno de campo Agronev}} \cite{agronev}
\begin{itemize}
\item Labores - Asignación de labores a parcelas, siembra, semilla tratada, aperos, imputación de costes
\item Abonado - Registro de Fertilización y Abonado. Composición de los Abonos, Forma de Abonado
\item Tratamientos - Tratamientos fitosanitarios en parcelas, eficacia, asesor, equipo de aplicación
\item Análisis de plaguicidas - Análisis de productos fitosanitarios, boletín de análisis, residuos detectados
\item Recolección - Registro de recolección y loteado. Asignación de venta directa, imputación de costes
\item Otros tratamientos - Aplicación de otros tratamientos fitosanitarios (Post-cosecha, Locales, Vehículos)
\item Costes - Imputación de gastos / costes a parcelas. Directos / Selectivos.
\item Gestión comercial - Compras, ventas, gastos, facturación, domiciliación bancaria SEPA, libro de fitos 
\end{itemize}
\end{itemize}