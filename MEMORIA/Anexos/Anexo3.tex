\chapter{Análisis} \label{c.analisis}



\section{Análisis de requisitos} 
\label{c.analisis.requisitos}

\subsection{Captura inicial de requisitos} 
\label{c.analisis.requisitos.iniciales}
\par El listado que se provee a continuación supone el resultado de la captura inicial de los requisitos sin filtrar ni categorizar. En la sección principal de la memoria correspondiente a este anexo (sección \ref{phytoscheme.requisitos}) se puede observar este listado tras haber sido priorizado mediante la \textit{Técnica \gls{moscow}}.

\begin{itemize}
\item Módulo de trazabilidad de los datos desde las fuentes originales hasta su visualización. 
\item Integración de más de dos fuentes de datos heterogéneas. 
\item Una memoria extensa y detallada.  
\item Mecanismo de control de versiones.
\item Desarrollo de tests adicionales en el proyecto de \textit{JHipster}
\item Desarrollo dirigido por un paradigma de inversión de independencias para conseguir un control centralizado.
\item Mecanismo de control de esfuerzos.
\item Genericidad en cuanto al soporte de integración de los datos de diferentes fuentes basado en un fichero de claves y valores. 
\item Ofrecer la infraestructura y las herramientas de configuración necesarias para una expansión futura del proyecto. 
\item Monitorizar y almacenar los procesos de recolección de los datos de entrada así como las rutas de su procesado.
\item Recolectar datos oficiales tanto de productos fitosanitarios autorizados de España como de las sustancias activas a nivel europeo. 
\item Rediseño y desarrollo en el lado del \textit{Front-End}.
\item Mecanismo de detección de errores e inconsistencias en los datos provenientes de diferentes fuentes.
\item Soporte para la búsqueda de registros (datos) desde la interfaz web. 
\item Almacenar la última versión de los datos en formato original y además mantener todas las versiones descargadas. 
\item Implementar un modelo de aplicación consistente, ejemplificando el ciclo de vida típico de los datos desde su recogida hasta su presentación visual. 
\end{itemize}

\subsection{La técnica \gls{moscow}} 
\label{c.analisis.requisitos.moscow}
Tal como se menciona en la sección \ref{phytoscheme.requisitos}, la técnica \textit{\gls{moscow}} es una técnica de priorización de requisitos usada en la gestión de proyectos, análisis de negocio y desarrollo de software con objetivo de llegar a un acuerdo común con los \textit{stakeholders} (integrantes del proyecto) sobre la importancia que se debería dar a cada requisito.
\par

\gls{moscow} dicta cuatro categorías mediante las que se pueden priorizar los requisitos de un sistema o proyecto: 

\begin{enumerate}
\item \textbf{Debe tener:} Son aquellos requisitos críticos para que el trabajo realizado durante un periodo de tiempo determinado (en nuestro caso desde ahora hasta junio 2017) sea considerado un éxito (\textit{TFG} aprobado). Si uno de estos requisitos no se incluye, el proyecto debera ser considerado un fallo (no se puede presentar el \textit{TFG}). \textit{Debe tener} en \textit{\gls{moscow}} se refiere a \textit{MUST}, que puede ser considerado un acrónimo de \textit{Minimum Usable SubseT}. En ese sentido se puede entender como la unión de los requisitos del producto mínimo viable con los requisitos legales (p.ej. documentación en forma de memoria de \textit{TFG}, cumplimiento \textit{LOPD}, etc.) y de seguridad (en el sentido de robustez y calidad de la solución) obligatorios o acordados.

\item \textbf{Debería tener:} Son aquellos que son importantes pero no necesarios para ser realizados durante el periodo de tiempo determinado. Pueden posponerse para ser realizados en el siguiente periodo. Son vitales pero no obligatorios, si no se implementan la solución es viable pero es doloroso no hacerlo.

\item \textbf{Podría tener:} Son aquellos que comparados con los anteriores son los menos deseados o tienen menor impacto. Hay que tenerlos controlados ya que sólo se podrán entregar si se dan las mejores condiciones (por ejemplo, el proyecto va más rápido de los esperado). Si hay algún riesgo en la entrega del proyecto estos requisitos serían los primeros en ser descartados.

\item \textbf{No tendrá:} Son aquellos que no van a ser entregados durante el periodo considerado. Se mantienen en esta lista de alcance para clarificar el alcance de la solución. Esto evita que informalmente sean introducidos más tarde. El objetivo de esta categoría es ayudar a mantener el foco en una solución estricta. 

\end{enumerate}

\section{Análisis de riesgos} \label{c.analisis.riesgos}
A continuación se exponen las diferentes fases del análisis de riesgos de manera detallada: 

\begin{enumerate}

\item \textbf{Identificación de los riesgos}

Se han intentado considerar el máximo número de riesgos y se han clasificado en diferentes categorías:

%FASE 1: Identificación de los riesgos
\begin{itemize}

\item \textbf{Riesgos globales} - ver tabla \ref{ries_glob}
\item \textbf{Riesgos tecnológicos} - ver tabla \ref{ries_tecno}
\item \textbf{Riesgos de alcance} - ver tabla \ref{ries_alcan}
\item \textbf{Riesgos de entorno de desarrollo} - ver tabla \ref{ries_entorno}

\bigskip
\begingroup
\renewcommand\arraystretch{1.3}

\begin{longtable}{l p{5cm} p{9cm}}
% header and footer information
\hline
\textbf{ID} & \textbf{Nombre} & \textbf{Explicación} \\
\hline
\endhead
\endfoot
RG\_1 & 
Plazos &
El proyecto no se finaliza para la convocatoria de junio, septiembre o diciembre.
 \\
RG\_2 & 
Fallo del equipo &
El equipo principal de desarrollo falla, se pierde o estropea. 
 \\
RG\_3 & 
Incorporación mercado &
El alumno se incorpora al mercado laboral durante el desarrollo del proyecto, a falta de varios meses de su finalización.
 \\
RG\_4 & 
Experiencia del alumno &
El alumno no dispone de los conocimientos y preparación suficiente para el desarrollo del proyecto. 
 \\
\hline
\caption{Riesgos globales del proyecto}\label{ries_glob}\\
\end{longtable}

\begin{longtable}{l p{5cm} p{9cm}}
% header and footer information
\hline
\textbf{ID} & \textbf{Nombre} & \textbf{Explicación} \\
\hline
\endhead
\endfoot
RT\_1 & 
Tecnología nueva &
Se trata de una tecnología nueva.
 \\
RT\_2 & 
Software no probado &
Se debe interactuar con software que no ha sido probado. 
 \\
RT\_3 & 
Interfaz especializada &
Es requerida una interfaz de usuario especializada.
 \\
RT\_4 & 
Componentes diferentes &
Se necesitan componentes de programa diferentes a los hasta ahora desarrollados.
 \\
RT\_5 & 
Rendimiento &
Se debe interactuar con un B.D. cuya funcionalidad y rendimiento no ha sido probada.
 \\
\hline
\caption{Riesgos tecnológicos del proyecto}\label{ries_tecno}\\
\end{longtable}

\begin{longtable}{l p{5cm} p{9cm}}
% header and footer information
\hline
\textbf{ID} & \textbf{Nombre} & \textbf{Explicación} \\
\hline
\endhead
\endfoot
RA\_1 & 
Tamaño estimado &
Tamaño estimado del proyecto
 \\
RA\_2 & 
Confianza en la estimación &
Confianza en la estimación
 \\
RA\_3 & 
Número de elementos &
Número de programas, archivos y transacciones
 \\
RA\_4 & 
Tamaño almacenamiento &
Tamaño de las bases de datos involucradas
 \\
RA\_5 & 
Número de usuarios &
Número de usuarios 
 \\
RA\_6 & 
Número de cambios &
Número de cambios en los requisitos
 \\
RA\_7 & 
Software reutilizado &
Cantidad de software utilizado
 \\
\hline
\caption{Riesgos de alcance del proyecto}\label{ries_alcan}\\
\end{longtable}

\begin{longtable}{l p{5cm} p{9cm}}
% header and footer information
\hline
\textbf{ID} & \textbf{Nombre} & \textbf{Explicación} \\
\hline
\endhead
\endfoot
RE\_1 & 
Gestión proyectos &
Hay herramientas de gestor de proyectos
 \\
RE\_2 & 
Gestión proceso desarrollo &
Hay herramientas de gestión del proceso de desarrollo
 \\
RE\_3 & 
Análisis y diseño &
Se usan métodos y herramientas específicas para el análisis y diseño 
 \\
RE\_4 & 
Generadores de código &
Hay generadores de código apropiados para la aplicación
 \\
RE\_5 & 
Pruebas &
Hay herramientas de pruebas apropiadas
 \\
RE\_6 & 
Gestión de configuración &
Hay herramientas de gestión de configuración apropiadas
 \\
RE\_7 & 
Base de datos  &
Se hace uso de una base de datos o repositorio centralizado
 \\
RE\_8 & 
Integración &
Están todas las herramientas de desarrollo integradas
 \\
RE\_9 & 
Formación &
Se ha proporcionado formación a todos los miembros del equipo de desarrollo
 \\
RE\_10 & 
Expertos &
Hay expertos a los cuales solicitar ayuda acerca de las herramientas
 \\
RE\_11 & 
Ayuda online &
Hay ayuda en línea y documentación disponible
 \\
RE\_12 & 
Diseño arquitectónico &
Se utiliza un  método específico para el diseño arquitectónico y de datos
 \\
RE\_13 & 
Métricas de calidad &
Se disponen métricas de calidad para todos los proyectos de software
 \\
RE\_14 & 
Métricas de productividad &
Se disponen de métricas de productividad
\\
\hline
\caption{Riesgos de entorno de desarrollo del proyecto}\label{ries_entorno}\\
\end{longtable}

\endgroup

\end{itemize}

\item \textbf{Análisis del riesgo} \par

Para esta fase se han empleado los tres medidores del riesgo: la probabilidad, el impacto y la aceptación: 

%FASE 2: Análisis del riesgo - INTRODUCCION
\begin{itemize}

\item{\textbf{Tabla para estimar la probabilidad de un riesgo:}} - ver tabla \ref{prob_riesgo}
\item{\textbf{Tabla para estimar el impacto de un riesgo:}} - ver tabla \ref{impacto_riesgo}
\item{\textbf{Tabla para estimar la aceptación de un riesgo:}} - ver tabla \ref{aceptacion_riesgo}


\bigskip

\begingroup
\renewcommand\arraystretch{1.3}
\begin{longtable}{l p{13cm}}
% header and footer information
\hline
\textbf{Valor} & \textbf{Descripción} \\
\hline
\endhead
\endfoot
Bajo (1) & 
La amenaza se materializa a lo sumo una vez cada año. 
 \\
Medio (2) & 
La amenaza se materializa a lo sumo una vez cada mes.
 \\
Alto (3) & 
La amenaza se materializa a lo sumo una vez cada semana.
 \\
\hline
\caption{Probabilidad de un riesgo}\label{prob_riesgo}\\
\end{longtable}

\begin{longtable}{l p{13cm}}
% header and footer information
\hline
\textbf{Valor} & \textbf{Descripción} \\
\hline
\endhead
\endfoot
Bajo (1) & 
El daño derivado de la materialización de la amenaza no tiene consecuencias relevantes para la consecución de los objetivos.  
 \\
Medio (2) & 
El daño derivado de la materialización de la amenaza tiene consecuencias reseñables para la consecución de los objetivos.
 \\
Alto (3) & 
El daño derivado de la materialización de la amenaza tiene consecuencias graves reseñables para la consecución de los objetivos.
 \\
\hline
\caption{Impacto de un riesgo}\label{impacto_riesgo}\\
\end{longtable}



\begin{longtable}{l p{13cm}}
% header and footer information
\hline
\textbf{Valor} & \textbf{Descripción} \\
\hline
\endhead
\endfoot
Riesgo $\leq$ & 
La organización considera el riesgo poco reseñable. 
 \\
Riesgo $\geq$ 4 & 
La organización considera el riesgo reseñable y debe proceder a su tratamiento.
 \\
\hline
\caption{Aceptación de un riesgo}\label{aceptacion_riesgo}\\
\end{longtable}

\endgroup
\end{itemize}
\par La aceptación es una medida delimitadora que define aquellos riesgos que son considerados aceptables y aquellos ante los que se deben tomar medidas. Para esta medida se ha establecido un criterio de aceptación de 4. Cualquier riesgo cuyo valor sea menor que 4 se considera aceptable y por tanto un riesgo poco reseñable, mientras que aquellos que se encuentran por encima de 4 se consideran reseñables y se debe proceder a su tratamiento. 
\par
El cálculo de la gravedad del riesgo y su aceptación se realiza de la siguiente manera: se multiplica la probabilidad por el impacto, y si dicho valor excede el límite del criterio de aceptación, el riesgo se considera reseñable. 
A continuación, en base a las métricas anteriores, se especifican los riesgos de la fase 1 en las mismas categorías iniciales. Se resaltan en rojo aquellos riesgos cuya aceptación supera el 4.

%FASE 2: Análisis del riesgo - TABLAS
\begin{itemize}

\item \textbf{Riesgos globales} - ver tabla \ref{ries_glob_valoracion}
\item \textbf{Riesgos tecnológicos} - ver tabla \ref{ries_tecno_valoracion}
\item \textbf{Riesgos de alcance} - ver tabla \ref{ries_alcan_valoracion}
\item \textbf{Riesgos de entorno de desarrollo} - ver tabla \ref{ries_entorno_valoracion}

\bigskip
\begingroup
\renewcommand\arraystretch{1.3}

\begin{longtable}{l p{5cm} ccc}
% header and footer information
\hline
\textbf{ID} & \textbf{Nombre} & \textbf{Probabilidad} & \textbf{Impacto} & \textbf{Riesgo} \\
\hline
\endhead
\endfoot
\textbf{RG\_1} & 
\textbf{Plazos} &
\textbf{2} &
\textbf{3} &
\textbf{6} 
 \\
RG\_2 & 
Fallo del equipo &
1 &
3 &
3 
 \\
RG\_3 & 
Incorporación mercado &
1 &
2 &
2 
 \\
RG\_4 & 
Experiencia del alumno &
2 &
2 &
4 
 \\
\hline
\caption{Valoración riesgos globales del proyecto}\label{ries_glob_valoracion}\\
\end{longtable}

\begin{longtable}{l p{5cm} ccc}
% header and footer information
\hline
\textbf{ID} & \textbf{Nombre} & \textbf{Probabilidad} & \textbf{Impacto} & \textbf{Riesgo} \\
\hline
\endhead
\endfoot
RT\_1 & 
Tecnología nueva &
3 &
1 &
3 
 \\
\textbf{RT\_2} & 
\textbf{Software no probado} &
\textbf{2} &
\textbf{3} &
\textbf{6} 
 \\
RT\_3 & 
Interfaz especializada &
1 &
1 &
1 
 \\
RT\_4 & 
Componentes diferentes &
3 &
1 &
3 
 \\
RT\_5 & 
Rendimiento &
2 &
2 &
4 
 \\
\hline
\caption{Valoración riesgos tecnológicos del proyecto}\label{ries_tecno_valoracion}\\
\end{longtable}

\begin{longtable}{l p{5cm} ccc}
% header and footer information
\hline
\textbf{ID} & \textbf{Nombre} & \textbf{Probabilidad} & \textbf{Impacto} & \textbf{Riesgo} \\
\hline
\endhead
\endfoot
\textbf{RA\_1} & 
\textbf{Tamaño estimado} &
\textbf{2} &
\textbf{3} &
\textbf{6} 
 \\
RA\_2 & 
Confianza en la estimación &
2 &
2 &
4 
 \\
\textbf{RA\_3} & 
\textbf{Número de elementos} &
\textbf{2} &
\textbf{3} &
\textbf{6} 
 \\
RA\_4 & 
Tamaño almacenamiento &
1 &
3 &
3 
 \\
RA\_5 & 
Número de usuarios &
1 &
3 &
3 
 \\
\textbf{RA\_6} & 
\textbf{Número de cambios} &
\textbf{2} &
\textbf{3} &
\textbf{6} 
 \\
RA\_7 & 
Software reutilizado &
1 &
1 &
1 
 \\
\hline
\caption{Valoración riesgos de alcance del proyecto}\label{ries_alcan_valoracion}\\
\end{longtable}

\begin{longtable}{l p{5cm} ccc}
% header and footer information
\hline
\textbf{ID} & \textbf{Nombre} & \textbf{Probabilidad} & \textbf{Impacto} & \textbf{Riesgo} \\
\hline
\endhead
\endfoot
RE\_1 & 
Gestión proyectos &
1 &
1 &
1 
 \\
RE\_2 & 
Gestión proceso desarrollo &
1 &
1 &
1 
 \\
RE\_3 & 
Análisis y diseño &
1 &
2 &
2 
 \\
RE\_4 & 
Generadores de código &
1 &
1 &
1 
 \\
RE\_5 & 
Pruebas &
2 &
2 &
4 
 \\
RE\_6 & 
Gestión de configuración &
2 &
2 &
4 
 \\
RE\_7 & 
Base de datos  &
1 &
3 &
3 
 \\
RE\_8 & 
Integración &
1 &
1 &
1 
 \\
\textbf{RE\_9} & 
\textbf{Formación} &
\textbf{2} &
\textbf{3} &
\textbf{6} 
 \\
RE\_10 & 
Expertos &
2 &
1 &
2 
 \\
RE\_11 & 
Ayuda online &
2 &
2 &
4 
 \\
RE\_12 & 
Diseño arquitectónico &
2 &
1 &
2 
 \\
RE\_13 & 
Métricas de calidad &
3 &
1 &
3 
 \\
RE\_14 & 
Métricas de productividad &
3 &
1 &
3 
\\
\hline
\caption{Valoración riesgos de entorno de desarrollo del proyecto}\label{ries_entorno_valoracion}\\
\end{longtable}

\endgroup

\end{itemize}

\item \textbf{Priorización de riesgos} \par
Esta fase incluye todos los riesgos, ordenados de mayor a menor severidad. Se resaltan en rojo los riesgos que habrá que considerar en un plan de defensa estratégico posterior:


\begingroup
\renewcommand\arraystretch{1.3}

\begin{longtable}{l p{5cm} c}
% header and footer information
\hline
\textbf{ID} & \textbf{Nombre} & \textbf{Riesgo} \\
\hline
\endhead
\endfoot
\textbf{RG\_1} & 
\textbf{Plazos} &
\textbf{6} 
 \\
\textbf{RT\_2} & 
\textbf{Software no probado} &
\textbf{6} 
 \\
\textbf{RA\_1} & 
\textbf{Tamaño estimado} &
\textbf{6} 
 \\
\textbf{RA\_6} & 
\textbf{Número de cambios} &
\textbf{6} 
 \\
\textbf{RE\_9} & 
\textbf{Formación} &
\textbf{6} 
 \\
 
RT\_5 & 
Rendimiento &
4 
 \\
RA\_2 & 
Confianza en la estimación &
4 
 \\
RE\_5 & 
Pruebas &
4 
 \\
RE\_11 & 
Ayuda online &
4 
 \\
RG\_2 & 
Fallo del equipo &
3 
 \\
RT\_1 & 
Tecnología nueva &
3 
 \\
RT\_4 & 
Componentes diferentes &
3 
 \\
RA\_4 & 
Tamaño almacenamiento &
3 
 \\
RA\_5 & 
Número de usuarios &
3 
 \\
RE\_7 & 
Base de datos &
3 
 \\
RE\_13 & 
Métricas de calidad &
3 
 \\
RE\_14 & 
Métricas de productividad &
3 
 \\
RG\_3 & 
Incorporación mercado &
2 
 \\
RE\_3 & 
Análisis y diseño &
2 
 \\
RE\_10 & 
Expertos &
2 
 \\
RE\_12 & 
Diseño arquitectónico &
2 
 \\
RT\_3 & 
Interfaz especializada &
1 
 \\
RT\_7 & 
Software reutilizado &
1 
 \\
RE\_1 & 
Gestión proyectos &
1 
 \\
RE\_2 & 
Gestión proceso desarrollo &
1 
 \\
RE\_4 & 
Generadores de código &
1 
 \\
RE\_8 & 
Integración &
1 
 \\
\hline
\caption{Priorización de riesgos del proyecto}\label{prioriz_riesg}\\
\end{longtable}

\endgroup
Como se puede apreciar, hay 5 riesgos cuyo factor de gravedad es preocupante y deben ser tratados acordemente: 
\begin{enumerate}
\item RG\_1. Riesgo global “Plazos”. Tiene que ver con el hecho de no acabar el proyecto dentro de los plazos establecidos para su defensa. Hay 2 fechas recomendables para su defensa, la primera en Junio de 2017 y la segunda en Septiembre de 2017. No obstante, se dispone de otra oportunidad en Diciembre de 2017, aunque sería la menos recomendable dado que supondría el retraso de la defensa y con ello la dificultad del estudiante de realizar otras actividades mientras tanto. A partir de Diciembre, la consecuencia sería volver a matricularse en el proyecto y aportar las tasas de la matrícula por segunda vez.
\item RT\_2. Riesgo de tecnologías “Software no probado”. Tiene que ver con la probabilidad de usar en el proyecto software que previamente no ha sido probado y pueda fallar. Obtuvo una valoración de gravedad de 6/6 puesto que si bien es cierto que todas las tecnologías han sido probadas individualmente y se sabe que funcionan bien, el proceso en su conjunto no ha sido probado. No se sabe si es viable o no.  
\item RA\_1. Riesgo de alcance “Tamaño estimado”. Tiene que ver con el hecho de que el proyecto resulte mucho más grande de lo estimado inicialmente, y por diferentes circunstancias no se llegue a finalizar.
\item RA\_6. Riesgo de alcance “Número de cambios”. Otro riesgo es el hecho de que los requisitos cambien constantemente, bien porque los clientes lo solicitan bien porque las propias tecnologías lo imponen. 
\item RE\_9. Riesgo de entorno de desarrollo “Formación”. Este riesgo trata con el hecho de que el alumno disponga de la formación necesaria y suficiente para lograr los objetivos propuestos.
\end{enumerate}

\item \textbf{Planificación de la gestión de riesgos} \par
En esta fase se recogen las conclusiones mitigadoras acerca de los riesgos "preocupantes" del proyecto, en relación a su factor de gravedad: 

\begin{enumerate}
\item RG\_1: Plazos - Como medida mitigante, el alumno deberá dedicar un horario de jornada completa a la realización del proyecto durante el verano del año 2017. 
\item RT\_2: Software no probado - La contrapartida y defensa de este riesgo es desarrollar o experimentar primero con una prueba de concepto para validar que las tecnologías en su conjunto funcionen correctamente. 
\item RA\_1: Tamaño estimado - La solución desde un principio debe definir bien el alcance y determinar aquellas cosas que formarán parte de la solución y aquellas que no lo harán. 
\item RA\_6: Número de cambios - El alumno y el profesor deben acordar al principio unos requisitos fijos que no podrán ser modificables, en conjunto con el hecho de definir claramente el alcance de la solución.
\item RE\_9: Formación - Para mitigar este riesgo, el alumno debe estar en constante aprendizaje, utilizando los manuales y tutoriales de las diferentes herramientas de las que va a hacer uso durante el proyecto. Además, el alumno tendrá a su disposición al director del proyecto para consultar dudas y a los diferentes foros tecnológicos de Internet. 
\end{enumerate}
\end{enumerate}

\section{Clientes potenciales} \label{b.clientes.potenciales}
\paragraph*{\textit{aGROSLab}} \cite{agroslab}
\begin{itemize}
\item Registro de Explotaciones - Las parcelas se presentan para su selección en un innovador formato de celdas, facilitando la visualización de todas sus características (identificación, cultivo, superficie, …), el acceso al visor GIS, la aplicación de filtros y su selección individual o en bloque.
\item Registro de Parcelas Agrícolas - permite cargar las parcelas que componen su explotación a partir de la información generada por el aplicativo de gestión de la Solicitud de Ayudas PAC, para la mayoría de las Comunidades Autónomas.
\item Compras de Productos Fitosanitarios - incorpora un registro de compras de productos fitosanitarios, en el que podrá archivar todas sus facturas de compra en formato PDF y a partir del cual podrá registrar los tratamientos realizados.
\item Registro de \gls{tratamiento}- filtra los productos autorizados para cada uno de sus cultivos, presenta las plagas para las que puede ser aplicado según la nomenclatura del \textit{MAPAMA} y le informa del tipo y rango de dosis que puede ser aplicada.
\item Registro de Comercialización de Cosecha - facilita el registro de la comercialización de su cosecha, presentado en formato de celdas el conjunto de parcelas de su explotación con un determinado cultivo e informando gráficamente de aquellas en las que puede existir un problema con los plazos de seguridad de un tratamiento fitosanitario.
\item Receta Fitosanitaria 
\item Visor GIS con Capas - permite al agricultor visualizar gráficamente las parcelas que componen su explotación y la información de cultivos y los tratamientos realizados. Una herramienta especialmente útil a la hora de identificar sus diferentes parcelas y tener una visión de conjunto de toda su explotación.
\item Unidades Homogéneas de Cultivo
\item Control de Consumos de Fitosanitarios - permite llevar el control de los productos (fitosanitarios y fertilizantes) adquiridos y los aplicados
\item Importaciones y Exportaciones 
\item Importaciones y Exportaciones 
\end{itemize}
\paragraph*{\textit{Agricolum}} \cite{agricolum}
\begin{itemize}
\item Web + APP móvil y tableta
\item Validación dosis cuaderno de campo
\item Gestión de personal y maquinaria
\item Informes personalizados y oficiales
\item Control por GPS
\item Control stock
\item Gestión económica
\item Rendimientos por campos y cultivos
\item Soporte telefónico y por internet
\item Gestión del cuaderno de campo
\item Aplicación conectada con los datos del Sigpac y fitosanitarios \textit{MAPAMA}
\item Sincronización de la información desde cualquier dispositivo
\item Vista de tiempo actual y previsión semanal
\item Ver histórico de todas las tareas realizadas
\item Saber en tiempo real el precio del mercado
\item Exportación de la información en otros formatos
\item Importación de los datos de la PAC
\item Generación del cuaderno de explotación oficial
\end{itemize}
\paragraph*{\textit{Cuaderno de campo Agronev}} \cite{agronev}
\begin{itemize}
\item Labores - Asignación de labores a parcelas, siembra, semilla tratada, aperos, imputación de costes
\item Abonado - Registro de Fertilización y Abonado. Composición de los Abonos, Forma de Abonado
\item Tratamientos - Tratamientos fitosanitarios en parcelas, eficacia, asesor, equipo de aplicación
\item Análisis de plaguicidas - Análisis de productos fitosanitarios, boletín de análisis, residuos detectados
\item Recolección - Registro de recolección y loteado. Asignación de venta directa, imputación de costes
\item Otros tratamientos - Aplicación de otros tratamientos fitosanitarios (Post-cosecha, Locales, Vehículos)
\item Costes - Imputación de gastos / costes a parcelas. Directos / Selectivos.
\item Gestión comercial - Compras, ventas, gastos, facturación, domiciliación bancaria SEPA, libro de fitos 
\end{itemize}


\section{Análisis de diseños alternativos} \label{c.analisis.disenyos}

Al igual que los requisitos, el diseño de la solución también sufrió constantes cambios. Inicialmente la propuesta de trazabilidad de la solución fue la que se puede observar en la figura \ref{fig:dis_1_sist}. \par

\begin{figure}[H]
    \centering
    \includegraphics[width=1\textwidth,height=5.5cm]{Imagenes/Dis_Fig_1}
    \caption{Diseño primitivo del sistema.}
    \label{fig:dis_1_sist}
\end{figure}
\par


Como se puede observar, inicialmente el concepto giraba alrededor de las tres tecnologías core: \textit{Apache Hadoop}, \textit{Apache Hive} y \textit{JHipster}. No se tuvo en consideración otras herramientas puesto que se pensaba que era suficiente para resolver el problema. Los datos en crudo, extraídos de la web del \textit{MAPAMA} o de otras fuentes heterogéneas, serían almacenados en \textit{Apache Hadoop}, en un nodo local mediante el sistema de ficheros \textit{HDFS}, y posteriormente sería \textit{Apache Hive} el encargado de procesarlos en su totalidad hasta conseguir almacenarlos en un esquema común. Además, la misma “base de datos” de \textit{Apache Hive} funcionaría como base de datos para la aplicación desarrollada con \textit{JHipster}, sirviendo en todo momento ese esquema único para la visualización del mismo en un navegador web. Este segundo diseño, conceptualmente fue la solución ideal para el problema planteado; no obstante, debido a que \textit{JHipster} no ofrece soporte para cambiar la base de datos con la que se construye la aplicación y mucho menos soporte para \textit{Apache Hive} o \textit{Apache Hadoop}, tras muchos intentos frustrados de conseguir esta conectividad directa, se optó por una solución diferente, alejada de este diseño ideal. La alternativa inmediata a este diseño fue la que se observa en la figura \ref{fig:dis_2_sist}. \par


\begin{figure}[H]
    \centering
    \includegraphics[width=1\textwidth,height=8cm]{Imagenes/Dis_Fig_2}
    \caption{Segunda iteración del diseño del sistema.}
    \label{fig:dis_2_sist}
\end{figure}
\par


En esta tercera iteración del diseño, se observa la evolución de la idea, condicionada por los problemas anteriormente mencionados. Se conservó la base de datos nativa de \textit{JHipster}, en este caso una base de datos \textit{MySQL} relacional. En ella se almacenaria únicamente la estructura final del esquema unificado, con los datos finales. Dichos datos tendrían que ser pasados desde \textit{Apache Hive} mediante una herramienta de transformación y transporte de datos. En este caso se usó \textit{Apache Sqoop}, una herramienta gratuita que permite transportar los datos desde \textit{Apache Hive} a \textit{MySQL}, puesto que ofrece soporte tanto para \textit{Apache Hadoop}, \textit{HDFS} y \textit{Apache Hive} como para \textit{MySQL}. \par
Una vez resuelto el problema de la visualización de los datos, lo próximo que se detectó fue esa necesidad de procesamiento de los datos en crudo antes de incluso exponerlos como un esquema relacional en \textit{Apache Hive}. Para eso, lo mejor era hacer uso de algún programa de procesado de ficheros y una de las mejores opciones aparentes fue \textit{Pentaho}, un programa completo de transformación de ficheros. Este venía con soporte para \textit{Apache Hadoop} y \textit{HDFS}, por lo que gracias a eso se pudo trabajar con ficheros directamente extraídos de \textit{Apache Hadoop}, y almacenarlos directamente en \textit{Apache Hadoop}. Los cambios se pueden observar en la figura \ref{fig:dis_3_sist}. \par


\begin{figure}[H]
    \centering
    \includegraphics[width=1\textwidth,height=7cm]{Imagenes/Dis_Fig_3}
    \caption{Tercera iteración del diseño del sistema.}
    \label{fig:dis_3_sist}
\end{figure}
\par

El siguiente cambio importante que sufrió el diseño fue la sustitución de \textit{Pentaho} por \textit{Talend}. Tal como se explica a lo largo de la sección \ref{implementacion.problemas}, \textit{Pentaho} no resultó ser la herramienta adecuada. Por lo tanto en su lugar, se utilizó \textit{Talend}. Otro cambio que se puede observar es que existen unos scripts \textit{Bash} capaces de extraer la información de la fuente (comando \textit{wget}), almacenarla en \textit{Apache Hadoop} como información en crudo y enviar dicha información a \textit{Talend} para ser procesada. 
\textit{Talend} después procesaría los datos y los almacenaría en \textit{Apache Hadoop} como \textit{Datos procesados}. \textit{Apache Hive} sería el encargado de recoger esos datos en formato relacional y \textit{Apache Sqoop} el responsable de su traslado hasta la base de datos \textit{MySQL} de \textit{JHipster}. Estos cambios se pueden apreciar en la figura \ref{fig:dis_4_sist}. Esta aproximación sería la última, la que mejor reflejaría el diseño final de la solución. Más detalles de su implementación se pueden encontrar en el capítulo \textit{Diseño} de esta memoria (Capítulo \ref{disenyo}).

\begin{figure}[H]
    \centering
    \includegraphics[width=1\textwidth,height=6cm]{Imagenes/Dis_Fig_4}
    \caption{Cuarta iteración del diseño del sistema.}
    \label{fig:dis_4_sist}
\end{figure}